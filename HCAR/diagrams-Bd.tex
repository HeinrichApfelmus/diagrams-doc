% diagrams-Bd.tex
\begin{hcarentry}[updated]{diagrams}
\report{Brent Yorgey}%11/10
\status{active development}
\makeheader

The diagrams library provides an embedded domain-specific language for
creating simple pictures and diagrams. Values of type \texttt{Diagram}
are built up in a compositional style from various primitives and
combinators, and can be rendered to a physical medium, such as a file
in PNG, PS, PDF, or SVG format.  The overall vision is for diagrams to
become a viable alternative to DSLs like MetaPost or Asymptote, but with
the advantages of being \emph{purely functional} and
\emph{embedded}.

For example, consider the following diagram to illustrate the 24
permutations of four objects:

%**<img width=770 src="./permutations.jpg">
%*ignore
\begin{center}
\includegraphics[width=0.47\textwidth]{html/permutations.jpg}
\end{center}
%*endignore

The diagrams library was used to create this diagram with very
little effort (about ten lines of Haskell, including the code to actually
generate permutations).  The source code for this diagram, as well as
other examples and further resources, can be found at
\url{http:/code.haskell.org/diagrams/}.

The library is currently undergoing a major rewrite, with the goal of
basing the entire library on a more flexible, semantically elegant
foundational core.  Good progress was made at the most recent
Philadelphia hackathon, and a preliminary release is in the works.
Planned features include pluggable rendering backends, support for
arbitrary vector spaces and for animation, more sophisticated paths
and path operations, and an xmonad-like core/contrib model for
incorporating user-submitted extension modules.

\FurtherReading
\begin{compactitem}
\item  \url{http://code.haskell.org/diagrams/} 
\item  \url{http://byorgey.wordpress.com/2009/09/24/diagrams-0-2-1-and-future-plans/} 
\item  \url{http://www.tug.org/metapost.html} 
\item  \url{http://asymptote.sourceforge.net/}
\end{compactitem}
\end{hcarentry}
